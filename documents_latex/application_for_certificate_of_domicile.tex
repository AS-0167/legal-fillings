\documentclass{article}
\usepackage[utf8]{inputenc}
\usepackage{amsmath}

\begin{document}

\begin{center}
\textbf{APPENDIX-VIII} \\
\textbf{FORM “P”} \\
\textbf{VIDE RULE 23, PAKISTAN CITIZENSHIP RULES – 1951} \\
\textbf{APPLICATION FOR CERTIFICATE OF DOMICILE PAKISTAN}
\end{center}

\vspace{0.5cm}

To \\
The District Coordination Officer,

\vspace{0.5cm}

I, \underline{\hspace{5cm}} S/O \underline{\hspace{5cm}} Hereby state that I was formerly

the resident of \underline{\hspace{5cm}} District \underline{\hspace{5cm}} Rev/Adm Punjab

have arrived in Tehsil \underline{\hspace{5cm}} day of \underline{\hspace{2cm}}. I have been

continuously residing in Pakistan for a period of \underline{\textbf{Since}} \underline{\hspace{2cm}} years \underline{\hspace{2cm}} months

immediately. Preceding this declaration and I hereby express my intention to

abandon my domicile of origin in \underline{\hspace{5cm}} and with a view to acquire the

remaining of my life.

I further affirm that I had not migrated to India \& returned to Pakistan between

the 1st March 1947 to the date of this application except on visa No. \underline{\hspace{5cm}}

dated \underline{\hspace{5cm}}.

Issued by the Pakistan Passport office at: \underline{\hspace{5cm}}

Other particulars are given below: -

Married/Single/Widow \underline{\hspace{5cm}}

Name of wife or husband: \underline{\hspace{5cm}}

Name of Children \& their ages: \underline{\hspace{5cm}}
\underline{\hspace{5cm}}
\underline{\hspace{5cm}}
\underline{\hspace{5cm}}

Trade \& Occupation: \underline{\hspace{5cm}}

Marks of Identification: \underline{\hspace{5cm}}

Do solemnly affirm that the above statement is true to the best of my knowledge and belief.

Attestation: \underline{\hspace{5cm}} \hspace{5cm} Signature: \underline{\hspace{5cm}}

Designation: \underline{\hspace{5cm}} \hspace{5cm} Place: \underline{\hspace{5cm}}

Place and Date: \underline{\hspace{5cm}} \hspace{5cm} Date: \underline{\hspace{5cm}}

\end{document}

